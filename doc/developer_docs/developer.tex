\input eplain

\font\titlefont = cmb24
\font\subtitlefont = cmr17
\font\subtitlett = cmtt12
\font\chapterfontbf = cmb17
\font\chapterfontrm = cmr17
\font\sectionfont = cmb14
\font\subsectionfont = cmb10
\font\bigrm = cmr12
\font\bigbf = cmb12
\font\bigit = cmti12

\newcount\chapterno
\newcount\sectionno

\def\chapter#1{\advance\chapterno by 1\sectionno = 0\vfill\break\null\vskip 1in
  {\noindent\hfill\chapterfontbf Chapter \the\chapterno\par}\medskip
  {\noindent\hfill\chapterfontrm #1\par}\medskip\hrule\bigskip}
  
\def\section#1{\advance\sectionno by 1
  \bigskip\noindent{\sectionfont \the\chapterno.\the\sectionno\quad #1}\medskip}

\def\subsection#1{\medskip\noindent{\subsectionfont #1}.}

\pageno=0
\footline{\ifnum\pageno=0\null\else\centerline{\the\pageno}\fi}

\null\vskip 2in
{\flushright\titlefont
The Complete Developer's Guide
to the MAGS System}
\vskip 0.5in
{\flushright\subtitlefont
Aaron W. Hsu
{\subtitlett arcfide@sacrideo.us}}\bigskip

{\flushright\subtitlefont
Karissa McKelvey
{\subtitlett krmckelv@indiana.edu}}\bigskip

{\flushright\subtitlefont
Joshua Cox
{\subtitlett joshcox@indiana.edu}}

\vskip 2in
\centerline{\subtitlefont 1st Edition}

\chapter{Introduction to MAGS}

\section{Welcome to MAGS}

\section{Design Goals}



\chapter{Architectural Design Overview}

\chapter{Application Server}

\section{Introduction to the Application Server}

\noindent
The MAGS Application Server is the glue that binds the various components that 
make MAGS into a single, integrated application. It is meant to be the single
unifying program, and as such, it is generally expected that most modifications
of MAGS will not need to touch the core Application Server proper. At the 
heart, the application server controls access to the database server. It manages
student rosters, grades, and runs the autograder on submissions it receives.
The application server is not a front-end, but middleware between front-end
clients like Guido and the database and autograder reports.
The server itself is written in Dyalog APL, which provides for a rapid, concise 
code base. It is neutral to the database server used and serves its clients 
through SOAP exchanges. Should the need for additional services arise, 
they are easily added using the simple and easy to use SAWS framework.

The server is deployed as a single, standalone workspace. This is much like a 
Smalltalk image for those familiar with that language. This workspace contains
a latent expression that automatically launches the server. When hacking the 
server, one usually opens the {\tt websrv.dws} workspace in a Dyalog session
and manages the code files (also called scripts) from there. The script files 
in the {\tt websrv/} directory---they have the {\tt .dyalog} extension---
correspond to namespaces in the workspace. In modern version of Dyalog, 
changes made to namespace scripts in the workspace will automatically be 
reflected in the script files after the user confirms a prompt.

Initializing, creating, managing, and removing the database is also done through
the {\tt websrv.dws} workspace. Commands for setting server options, as well as
starting and stopping the server also exist in the workspace. As you can see, 
nearly everything is managed through the application server. Each of these 
commands is documented in more detail further on, but this discussion gives you 
a few hints to get your bearings.
To begin delving deeper, we will start at the ends, the database schema and 
the Web Service interface, and then move to the middle, describing the methods 
and variables of the various namespaces in detail.

\section{Database Schema and Approach}

\section{Web Services}

\section{Public Interface}

\section{Workspace Overview}

\subsection{Reading the code}

\section{The Main Namespace}

\section{The WS Namespace}

\chapter{Guido Manual Commenting Interface}

\chapter{Submission Validation}

\chapter{Interfacing with Autograders or Writing Your Own}

\chapter{Scheme Autograder}

\chapter{Java Autograder}

\chapter{APL Autograder}

\chapter{Python Autograder}

\chapter{Quick Reference Material}

\bye
