\input eplain

\font\titlefont = cmb24
\font\subtitlefont = cmr17
\font\subtitlett = cmtt12
\font\chapterfontbf = cmb17
\font\chapterfontrm = cmr17
\font\sectionfont = cmb12
\font\subsectionfont = cmb10
\font\bigrm = cmr12
\font\bigbf = cmb12
\font\bigit = cmti12

\newcount\chapterno
\newcount\sectionno

\def\chapter#1{\advance\chapterno by 1\sectionno = 0\vfill\break\null\vskip 1in
  {\noindent\hfill\chapterfontbf Chapter \the\chapterno\par}\medskip
  {\noindent\hfill\chapterfontrm #1\par}\medskip\hrule\bigskip}
  
\def\section#1{\advance\sectionno by 1
  \bigskip\noindent{\sectionfont \the\chapterno.\the\sectionno\quad #1}\medskip}

\def\subsection#1{\medskip\noindent{\subsectionfont #1}.}

\pageno=0
\footline{\ifnum\pageno=0\null\else\centerline{\the\pageno}\fi}

\null\vskip 2in
{\flushright\titlefont
The Complete Developer's Guide
to the MAGS System}
\vskip 0.5in
{\flushright\subtitlefont
Aaron W. Hsu
{\subtitlett arcfide@sacrideo.us}}\bigskip

{\flushright\subtitlefont
Karissa McKelvey
{\subtitlett krmckelv@indiana.edu}}\bigskip

{\flushright\subtitlefont
Joshua Cox
{\subtitlett joshcox@indiana.edu}}

\vskip 2in
\centerline{\subtitlefont 1st Edition}

\chapter{Introduction to MAGS}

\section{Welcome to MAGS}

\section{Design Goals}



\chapter{Architectural Design Overview}

\chapter{Application Server}

\section{Introduction to the Application Server}

\noindent
The MAGS Application Server is the glue that binds the various components that 
make MAGS into a single, integrated application.  It is meant to be the single
unifying program, and as such, it is generally expected that most modifications
of MAGS will not need to touch the core Application Server proper.  At the 
heart, the application server controls access to the database server.  It manages
student rosters, grades, and runs the autograder on submissions it receives.
The application server is not a front-end, but middleware between front-end
clients like Guido and the database and autograder reports.
The server itself is written in Dyalog APL, which provides for a rapid, concise 
code base.  It is neutral to the database server used and serves its clients 
through SOAP exchanges.  Should the need for additional services arise, 
they are easily added using the simple and easy to use SAWS framework.

The server is deployed as a single, standalone workspace.  This is much like a 
Smalltalk image for those familiar with that language.  This workspace contains
a latent expression that automatically launches the server.  When hacking the 
server, one usually opens the {\tt websrv.dws} workspace in a Dyalog session
and manages the code files (also called scripts) from there.  The script files 
in the {\tt websrv/} directory---they have the {\tt .dyalog} extension---
correspond to namespaces in the workspace.  In modern version of Dyalog, 
changes made to namespace scripts in the workspace will automatically be 
reflected in the script files after the user confirms a prompt.

Initializing, creating, managing, and removing the database is also done through
the {\tt websrv.dws} workspace.  Commands for setting server options, as well as
starting and stopping the server also exist in the workspace.  As you can see, 
nearly everything is managed through the application server.  Each of these 
commands is documented in more detail further on, but this discussion gives you 
a few hints to get your bearings.
To begin delving deeper, we will start at the ends, the database schema and 
the Web Service interface, and then move to the middle, describing the methods 
and variables of the various namespaces in detail.

\section{Database Schema and Approach}

\noindent
The application server uses a fairly simple database schema.
Our relational model is believed to be in BC normal form and 
should have no redundancy of data storage.
Moreover, we explicitely do not permit {\tt null} data in any attribute.
Most relations have only a simple natural key and we avoid any 
artifical keys. Each table may be an entity, a relation, or mixed. 
Mixed tables are entities with certain bijective relations merged into them.
All fields contain atomic data.
We make heavy use of foreign and normal key constraints, 
but few other constraints are necessary for our model.

This chapter documents the database schema in detail and should be 
considered the authoritative reference. 
Inconsistencies between this document and the application server code should be 
treated as serious bugs. The following is a short summary of each table's 
purpose; each heading in this section focuses on detailing one of these tables.

$$\vbox{
  \offinterlineskip
  \halign{
    \strut \quad #  & \vrule\quad #  & \vrule\quad # \cr
    {\bf Entities}  & {\bf Relations}  & {\bf Mixed}\cr
    \noalign{\hrule}    
    Instructor  & Validates  & Submitter \cr
    Groups  & Teaches  & Submission \cr
    Assignment  & HasType  & \cr
    Problem  & HasDeadline  & \cr
    Category & BelongsTo  & \cr
    Comment & Contains  & \cr
    Deadline & Collaborated  & \cr
    Validator & CommentsOn  & \cr
  }
}$$

\subsection{Instructor Entity}
This entity holds the set of all non-students, 
whether teachers, assistents, or just graders.
A member of this entity is allowed to grade assignment submissions
based on what groups they may ``teach.''
Thus, not all instructors may grade all submissions.

$$\vbox{
  \offinterlineskip
  \halign{
    \strut \quad {\tt #}  & \vrule\quad #  & \vrule\quad # \cr
    {\bf Field Name}  & {\bf Type}  & {\bf Foreign Key}\cr
    \noalign{\hrule}    
    networkid  & Fixed String  & --- \cr
    firstname  & Variable String  & --- \cr
    lastname  & Variable String  & --- \cr
  }
}$$

{\noindent\flushleft
Natural Key: {\tt networkid}
Participates in these relations: Teaches}

\subsection{Group Entity}
This entity currently records the names of various groups,
which may correspond to either internal groups for a class
or to class sections or any other division.
It serves as the divider by which submitters are grouped.
Assignments are also assigned to specific groups.

$$\vbox{
  \offinterlineskip
  \halign{
    \strut \quad {\tt #}  & \vrule\quad #  & \vrule\quad # \cr
    {\bf Field Name}  & {\bf Type}  & {\bf Foreign Key}\cr
    \noalign{\hrule}    
    name  & Variable String  & --- \cr
  }
}$$

{\noindent\flushleft
Natural Key: {\tt name}
Participates in these relations: Submitter.member, Teaches, and BelongsTo}

\subsection{Assignment Entity}
An assignment is a named collection of problems.
Problem Set might be another good name for it.
Assignments are given to groups of students and have auto-graders
associated with them in the form of validators and problem auto-graders.
When a submitter submits a submission, 
it is submitted for a particular assignment.
Assignments may be reused or used for particular groups
(that is, used for more than one group). 
Assignments can be categorized.

$$\vbox{
  \offinterlineskip
  \halign{
    \strut \quad {\tt #}  & \vrule\quad #  & \vrule\quad # \cr
    {\bf Field Name}  & {\bf Type}  & {\bf Foreign Key}\cr
    \noalign{\hrule}    
    name  & Variable String  & --- \cr
  }
}$$

{\noindent
Natural Key: {\tt name}\par\noindent
Participates in these relations: Submission.submittedfor, Validates,
HasType, HasDeadline, BelongsTo, Contains, Collaborated, CommentsOn\par}

\subsection{Problem Entity}
Problems are the central entity in MAGS. 
They represent the unit of grading and most everything else is based on 
the concept of grading a problem.
Autograders are applied to individual problems. 
They are composed of a name, a description, a test suite, and a solution.

$$\vbox{
  \offinterlineskip
  \halign{
    \strut \quad {\tt #}  & \vrule\quad #  & \vrule\quad # \cr
    {\bf Field Name}  & {\bf Type}  & {\bf Foreign Key}\cr
    \noalign{\hrule}    
    name  & Variable String  & --- \cr
    description  & Variable String  & --- \cr
    testsuite  & Variable String  & --- \cr
    solution  & Variable String & --- \cr
    grader & Variable String & --- \cr
    grader\_params & Variable String & --- \cr
  }
}$$

{\noindent
Natural Key: {\tt name}\par\noindent
Participates in these relations: Contains, CommentsOn\par}

\subsection{Submitter Entity/Relation}
This mixed entity/relation captures the concept
of a student or submitter of solutions.
Each submitter belongs to a group.
Currently, a submitter may be a member of only one group,
but this may change in the future.

$$\vbox{
  \offinterlineskip
  \halign{
    \strut \quad {\tt #}  & \vrule\quad #  & \vrule\quad # \cr
    {\bf Field Name}  & {\bf Type}  & {\bf Foreign Key}\cr
    \noalign{\hrule}    
    networkid  & Fixed String  & --- \cr
    firstname  & Variable String  & --- \cr
    lastname  & Variable String  & --- \cr
    member  & ---  & {\tt Group.name} \cr
  }
}$$

{\noindent
Natural Key: {\tt networkid}\par\noindent
Participates in these relations: Submission.owner, Collaborated, CommentsOn \par}

\subsection{Submission Entity/Relation}
A submission is a proposed solution to a given assignment
submitted by a submitter. It has a submission date associated with 
it and may be marked as an appeal, meaning it is a resubmission
of a previous submission.
The autograder report is attached to the submission directly 
because submissions are intended to be graded immediately on submission
and should have only a single report.

$$\vbox{
  \offinterlineskip
  \halign{
    \strut \quad {\tt #}  & \vrule\quad #  & \vrule\quad # \cr
    {\bf Field Name}  & {\bf Type}  & {\bf Foreign Key}\cr
    \noalign{\hrule}    
    submissionid & Serial (Integer) & --- \cr
    owner  & ---  & {\tt Submitter.networkid} \cr
    submittedfor  & ---  & {\tt Assignment.name} \cr
    date  & Date  & --- \cr
    isappeal  & Boolean  & --- \cr
    code  & Variable String  & --- \cr
    report  & XML  & --- \cr
  }
}$$

{\noindent
Natural Key: {\tt owner}, {\tt submittedfor}, {\tt date}\par\noindent
Participates in these relations: Collaborated, CommentsOn \par}

\subsection{Category Entity}
Categories are simple organizers for assignments. 
They are largely informational.

$$\vbox{
  \offinterlineskip
  \halign{
    \strut \quad {\tt #}  & \vrule\quad #  & \vrule\quad # \cr
    {\bf Field Name}  & {\bf Type}  & {\bf Foreign Key}\cr
    \noalign{\hrule}    
    name  & Variable String  & --- \cr
  }
}$$

{\noindent
Natural Key: {\tt name}\par\noindent
Participates in these relations: HasType \par}

\subsection{Deadline Entity}
A deadline is some date with a type tag that can be used to track certain
features of a submission, such as whether it was submitted on time.
Assignments may have many such deadlines. While the natural key is 
the {\tt type} and {\tt date}, we use a {\tt deadlineid} field to make 
matching in foreign keys doable.

$$\vbox{
  \offinterlineskip
  \halign{
    \strut \quad {\tt #}  & \vrule\quad #  & \vrule\quad # \cr
    {\bf Field Name}  & {\bf Type}  & {\bf Foreign Key}\cr
    \noalign{\hrule}
    deadlineid & Serial (Integer) & --- \cr
    type  & Variable String  & --- \cr
    date  & Date  & --- \cr
  }
}$$

{\noindent
Natural Key: {\tt type}, {\tt date}\par\noindent
Participates in these relations: HasDeadline \par}

\subsection{Validator Entity}
A validator is a program which will be run to verify that a submission 
is acceptable before running the autograder on it.
A failed validator results in rejecting a submission.
Unlike autograders, validators are associated with assignments.

$$\vbox{
  \offinterlineskip
  \halign{
    \strut \quad {\tt #}  & \vrule\quad #  & \vrule\quad # \cr
    {\bf Field Name}  & {\bf Type}  & {\bf Foreign Key}\cr
    \noalign{\hrule}    
    name  & Variable String  & --- \cr
    command  & Variable String  & --- \cr
  }
}$$

{\noindent
Natural Key: {\tt name}\par\noindent
Participates in these relations: Validates \par}

\subsection{Comment Entity}
A comment is just some text.
We have an entity for it because Guido has a concept of reusing comments.

$$\vbox{
  \offinterlineskip
  \halign{
    \strut \quad {\tt #}  & \vrule\quad #  & \vrule\quad # \cr
    {\bf Field Name}  & {\bf Type}  & {\bf Foreign Key}\cr
    \noalign{\hrule}    
    text  & Variable String  & --- \cr
  }
}$$

{\noindent
Natural Key: {\tt text}\par\noindent
Participates in these relations: CommentsOn \par}

\subsection{Validates Relation}
Indicates that a given validator should be used to validate 
submissions for a particular assignment. It also prescribes any 
additional parameters for use when calling the validator.

$$\vbox{
  \offinterlineskip
  \halign{
    \strut \quad {\tt #}  & \vrule\quad #  & \vrule\quad # \cr
    {\bf Field Name}  & {\bf Type}  & {\bf Foreign Key}\cr
    \noalign{\hrule}    
    validator  & ---  & {\tt Validator.name} \cr
    assignment  & ---  & {\tt Assignment.name} \cr
    params  & Variable String  & --- \cr
  }
}$$

{\noindent
Natural Key: {\tt validator}, {\tt assignment}, {\tt params}\par}

\subsection{Teaches Relation}
Binds or associates a member of Instructors as assigned to or grading a group.

$$\vbox{
  \offinterlineskip
  \halign{
    \strut \quad {\tt #}  & \vrule\quad #  & \vrule\quad # \cr
    {\bf Field Name}  & {\bf Type}  & {\bf Foreign Key}\cr
    \noalign{\hrule}    
    teacher  & ---  & {\tt Instructor.networkid} \cr
    group  & ---  & {\tt Group.name} \cr
  }
}$$

{\noindent
Natural Key: {\tt teacher}, {\tt group}\par}

\subsection{HasType Relation}
Associates an assignment with a category. 
Assignments may belong to multiple categories.

$$\vbox{
  \offinterlineskip
  \halign{
    \strut \quad {\tt #}  & \vrule\quad #  & \vrule\quad # \cr
    {\bf Field Name}  & {\bf Type}  & {\bf Foreign Key}\cr
    \noalign{\hrule}    
    assignment  & ---  & {\tt Assignment.name} \cr
    category  & ---  & {\tt Category.name} \cr
  }
}$$

{\noindent
Natural Key: {\tt assignment}, {\tt category}\par}

\subsection{HasDeadline Relation}
Assigns or gives an assignment a deadline,
where deadlines have types and dates.

$$\vbox{
  \offinterlineskip
  \halign{
    \strut \quad {\tt #}  & \vrule\quad #  & \vrule\quad # \cr
    {\bf Field Name}  & {\bf Type}  & {\bf Foreign Key}\cr
    \noalign{\hrule}    
    assignment  & ---  & {\tt Assignment.name} \cr
    deadline & --- & {\tt Deadline.deadlineid} \cr
  }
}$$

{\noindent
Natural Key: {\tt assignment}, {\tt type}, {\tt date}\par}

\subsection{BelongsTo Relation}
Associates an assignment to a group.
Assignments may belong to more than one group.

$$\vbox{
  \offinterlineskip
  \halign{
    \strut \quad {\tt #}  & \vrule\quad #  & \vrule\quad # \cr
    {\bf Field Name}  & {\bf Type}  & {\bf Foreign Key}\cr
    \noalign{\hrule}    
    assignment  & ---  & {\tt Assignment.name} \cr
    group  & ---  & {\tt Group.name} \cr
  }
}$$

{\noindent
Natural Key: {\tt assignment}, {\tt group}\par}

\subsection{Contains Relation}
Assigns a problems to an assignment with a given problem number.
Assignments should not reuse a problem,
but multiple assignments may use the same problem.

$$\vbox{
  \offinterlineskip
  \halign{
    \strut \quad {\tt #}  & \vrule\quad #  & \vrule\quad # \cr
    {\bf Field Name}  & {\bf Type}  & {\bf Foreign Key}\cr
    \noalign{\hrule}    
    assignment  & ---  & {\tt Assignment.name} \cr
    problem  & ---  & {\tt Problem.name} \cr
    number  & Variable String  & --- \cr
  }
}$$

{\noindent
Natural Key: {\tt assignment}, {\tt problem}\par}

\subsection{Collaborated Relation}
Indicates other students with whom the submitter of a submission
collaborated with during the completion of an assignment.
While other collaborations may have occurred, only submitter 
collaborations are currently tracked.

$$\vbox{
  \offinterlineskip
  \halign{
    \strut \quad {\tt #}  & \vrule\quad #  & \vrule\quad # \cr
    {\bf Field Name}  & {\bf Type}  & {\bf Foreign Key}\cr
    \noalign{\hrule}   
    submission & --- & {\tt Submission.submissionid} \cr 
    collaborator  & ---  & {\tt Submitter.networkid} \cr
  }
}$$

{\noindent
Natural Key: {\tt submission}, {\tt collaborator}\par}

\subsection{CommentsOn Relation}
An instructor comments on a submission by taking a comment 
and associating it with a start and end position somewhere in the submission. 
A comment may be used in multiple places in an assignment. 
Additionally a comment is marked as relating to a specific problem.
Please note the unmentioned constraint here, that the problem that is commented 
on should be one of the problems contained by the assignment for which the 
submission was submitted. If this is not the case, then who knows what could 
happen.

$$\vbox{
  \offinterlineskip
  \halign{
    \strut \quad {\tt #}  & \vrule\quad #  & \vrule\quad # \cr
    {\bf Field Name}  & {\bf Type}  & {\bf Foreign Key}\cr
    \noalign{\hrule}    
    comment  & ---  & {\tt Comment.text} \cr
    submission & --- & {\tt Submission.submissionid} \cr
    problem  & ---  & {\tt Problem.name} \cr
    start  & Int  & --- \cr
    end  & Int  & --- \cr
  }
}$$

{\noindent
Natural Key: {\tt comment}, {\tt submission}, 
{\tt problem}, {\tt start}, {\tt end}\par}

\section{Web Services}

\section{Public Interface}

\section{Workspace Overview}

\subsection{Reading the code}

\section{The Main Namespace}

\section{The WS Namespace}

\chapter{Guido Manual Commenting Interface}

\chapter{Submission Validation}

\chapter{Interfacing with Autograders or Writing Your Own}

The main MAGS application server has no built-in auto-grading support. 
Auto-grading is very specific to the domain and especially the language
being used. To make MAGS able to handle many different environments, 
we have no specific interface for auto-graders except by standard input and 
standard output. Auto-graders are invoked as shell commands with specific, 
standard arguments. The test results should be printed to standard output. The 
output is in XML format. Most programming languages that may benefit from MAGS
have mature XML capabilities so this should not be an imposition. The rest of 
this chapter focuses on these two interfaces: the command-line format and 
the XML schema. Once an auto-grader has been designed according to these 
specifications, it need only be installed in the binary path of the application
server before it may be used.

\section{Command-line Interface}

\noindent
The command-line interface described here gives the arguments with which each 
instance of the auto-grader is called. In particular, each problem in an 
assignment will spawn one instance of the auto-grader. Each invocation is 
expected to be referentially transparent, meaning that multiple runs of an 
auto-grader with the same arguments should produce the same output. One should 
not rely on hidden state to get reasonable results. The pattern fo arguments
looks like this:

\medskip\verbatim
<grader> : test-suite solution submission [extra ...]
|endverbatim\medskip

\noindent
Each problem is associated with a {\tt test-suite} and a {\tt solution}, so 
these will change on a per problem basis. The {\tt submission} is provided 
as the unique student submission. It is tied to the code field of the student
submission database. The {\tt extra} fields are given on a per problem basis
to provide more information to the auto-grader if necessary. If an error 
occurs during grading, this should be reported through the return value of 
the program according to the standard operating system conventions.

\section{XML Result Schema}

\noindent
Auto-graders return grading results via standard output in an XML format 
documented here. In particular, grading results are divided into tests that are 
expected to either pass or fail. The result of a test indicates what the 
expected result was as well as the actual result of the test. In addition to 
the pass/fail result of the test and the name of the test, there are 
an arbitrary number of properties that can store additional information. 
The actual properties used in an auto-grader are chosen by the specific
auto-grader and may not even use the same set of properties for each test in a 
given run. That is, all properties are optional. Tests may be grouped 
into test groups, but need not be. Test groups may also nest. All results 
are contained inside a root {\tt grading-results} node. What follows is a 
description of the nodes and their attributes.

\medskip
{\flushleft 
Node Name: {\tt grading-results}
Attributes: ---
Valid Child Nodes: {\tt test-group}, {\tt test-result}
Description: Root node of the auto-grader output.
}\medskip

{\flushleft
Node Name: {\tt test-group}
Attributes: {\tt name}
Valid Child Nodes: {\tt test-group}, {\tt test-result}
Description: Groups together a set of testing results.
}\medskip

{\flushleft
Node Name: {\tt test-result}
Attributes: {\tt name}, {\tt result}, {\tt expected}
Valid Child Nodes: {\tt test-property}
Description: Gives the result of a single test, its pass or fail status, and the 
name of the test.
}\medskip

{\flushleft
Node Name: {\tt test-property}
Attributes: {\tt name}
Valid Child Nodes: Character data
Description: Gives a single property of a test result. The value of the property 
is given as data in the property, while the name is given as an attribute.
}\medskip

\section{Handling Errors}

\noindent
If a critical error in grading occurs, the auto-grader is not expected to 
produce valid result data provided this situation is adequately reported in the 
return code. In this case, the output thus far will be used in the error but 
will not be used as report data. 

\chapter{Scheme Autograder}

\chapter{Java Autograder}

\chapter{APL Autograder}

\chapter{Python Autograder}

\chapter{Quick Reference Material}

\bye
